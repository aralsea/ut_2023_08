\documentclass[uplatex,a4paper,dvipdfmx]{jsarticle}
\usepackage{amsfonts, amsmath, amsthm, amssymb}
\usepackage{mathtools}
\usepackage{tikz-cd}

\renewcommand{\theenumi}{\arabic{enumi}}
\renewcommand{\labelenumi}{(\theenumi) }
\providecommand{\MR}{\relax\ifhmode\unskip\space\fi MR }
%% Theorems
\theoremstyle{plain}
\newtheorem{theorem}{Theorem}[section]
\newtheorem{conjecture}[theorem]{Conjecture}
\newtheorem{lemma}[theorem]{Lemma}
\newtheorem{proposition}[theorem]{Proposition}
\newtheorem{corollary}[theorem]{Corollary}

\theoremstyle{definition}
\newtheorem{definition}[theorem]{Definition}
\newtheorem{example}[theorem]{Example}
\newtheorem{remark}[theorem]{Remark}

%% Some operators
\DeclareMathOperator{\Hom}{\mathrm{Hom}}
\DeclareMathOperator{\Tor}{\mathrm{Tor}}
\DeclareMathOperator{\CHom}{\mathcal{H}\!\mathit{om}}
\DeclareMathOperator{\CTor}{\mathcal{T}\!\mathit{or}}
\DeclareMathOperator{\Auteq}{\mathrm{Auteq}}
\DeclareMathOperator{\Cone}{\mathrm{Cone}}
\DeclareMathOperator{\ev}{\mathrm{ev}}
\DeclareMathOperator{\id}{\mathrm{id}}
\DeclareMathOperator{\depth}{\mathrm{depth}}
\DeclareMathOperator{\Pic}{\mathrm{Pic}}
\DeclareMathOperator{\MCG}{\mathrm{MCG}}
\DeclareMathOperator{\PMCG}{\mathrm{PMCG}}
\DeclareMathOperator{\RHom}{\mathrm{RHom}}
\DeclareMathOperator{\Ker}{\mathrm{Ker}}
\DeclareMathOperator{\Image}{\mathrm{Im}}
\DeclareMathOperator{\Aut}{\mathrm{Aut}}
\DeclareMathOperator{\Inn}{\mathrm{Inn}}
\DeclareMathOperator{\Out}{\mathrm{Out}}
\DeclareMathOperator{\Supp}{\mathrm{Supp}}
\DeclareMathOperator{\SL}{\mathrm{SL}}
\DeclareMathOperator{\Spec}{\mathrm{Spec}}
\DeclareMathOperator{\Perf}{\mathrm{Perf}}
\DeclareMathOperator{\NS}{\mathrm{NS}}
\DeclareMathOperator{\Ext}{\mathrm{Ext}}
\DeclareMathOperator{\Hilb}{\mathrm{Hilb}}
\DeclareMathOperator{\res}{\mathrm{res}}
\DeclareMathOperator{\Ch}{\mathrm{Ch}}
\DeclareMathOperator{\coh}{\mathrm{coh}}


\newcommand{\nc}{\newcommand}

%% Calligraphic letters

\nc{\cF}{{\mathcal{F}}}
\nc{\cG}{{\mathcal{G}}}
\nc{\cH}{{\mathcal{H}}}

\nc{\cO}{{\mathcal{O}}}
\nc{\cU}{{\mathcal{U}}}
\nc{\cW}{{\mathcal{W}}}

%% Blackboard letters
\nc{\bA}{{\mathbb{A}}}
\nc{\bC}{{\mathbb{C}}}
\nc{\bP}{{\mathbb{P}}}
\nc{\bQ}{{\mathbb{Q}}}
\nc{\bZ}{{\mathbb{Z}}}


\title{代数多様体の導来圏の半捻り関手}
\author{荒井 勇人}
\date{\today}
\begin{document}
\maketitle

\begin{abstract}
	代数多様体$X$の連接層の導来圏$D^b(X)$の自己同値を構成する重要な手法のひとつが、SeidelとThomasにより導入された球面対象$E \in D^b(X)$に沿う捻り関手$T_E \in \mathrm{Auteq} D^b(X)$である。これはホモロジー的ミラー対称性のもとでシンプレクティック多様体上のLagrange球面に沿うDehn捻りに対応する。

	本稿では捻り関手の変種として、ミラー対称性のもとで穴あき曲面の半捻りに対応する$D^b(X)$の自己同値(半捻り関手)を構成し、それを用いて楕円曲面$S$の導来圏の自己同値群$\Auteq D^b(S)$の構造を調べる。
\end{abstract}

\section{導入}
\subsection{代数多様体の導来圏と自己同値群}
$X$を$\bC$上の代数多様体とし、$X$上の連接層のなすアーベル圏を$\coh X$、その有界な(コチェイン)複体のなすアーベル圏を$\Ch^b(\coh X)$とする。
複体の間の射$f \colon E \to F$は、複体のコホモロジーに誘導する射$H^n(f) \colon H^n(E) \to H^n(F)$が全ての$n \in \bZ$について同型であるとき擬同型と呼ばれる。
$X$の導来圏$D^b(X)$とは、$\Ch(\coh X)$を擬同型全体で局所化して得られる圏であり、自然に$\bC$線形な三角圏の構造を持つ。
また$D^b(X)$の自己同値群$\Auteq D^b(X)$を、$\bC$線形で三角圏の構造を保つような$D^b(X)$の自己同値関手の同型類がなす群とする。
$D^b(X)$の自己同値として、$X$の自己同型$f$による逆像$f^*$、$X$上の直線束$L$によるテンソル積$(-)\otimes L$、および三角圏のシフト関手$(-)[1]$が常に存在し、これらの生成する$\Auteq D^b(X)$の部分群
\begin{equation}
	A(X) \coloneq \Aut(X) \ltimes \Pic(X) \times \bZ[1]
\end{equation}
の元は標準的な自己同値と呼ばれる。
自己同値群についてのもっとも基本的な結果は次のBondalとOrlovによる定理である。
\begin{theorem}[\cite{MR1818984}]
	$X$を非特異射影多様体とし、$X$の標準束$\omega_X$またはその双対束$\omega_X^\vee$が豊富な直線束だと仮定する。
	このとき$A(X) = \Auteq D^b(X)$である。
\end{theorem}
この結果により、非自明な包含関係$A(X) \subsetneq \Auteq D^b(X)$が生じうる一番単純な状況は$X$が楕円曲線の場合となる。
このときの$\Auteq D^b(X)$はOrlovによるアーベル多様体についての結果を適用すると以下のようになる。
\begin{theorem}[\cite{MR1921811}]
	$X$を楕円曲線とすると、以下の完全列が存在する。
	\begin{equation}\label{eq:autoequivalence_of_elliptic_curve}
		1 \to \Aut{X} \ltimes \Pic^{0}(X)\times \bZ[2] \to \Auteq{D^b(X)} \xrightarrow{\theta} \SL(2, \bZ) \to 1
	\end{equation}
	ここで$\theta$は偶数次のコホモロジー群$H^0(X, \bZ) \oplus H^0(X, \bZ) \cong \bZ^2$への作用で与えられる。
\end{theorem}
これにより$A(X)$の像を計算すると、$\theta((-)[1]) = \begin{pmatrix}
		-1 & 0  \\
		0  & -1 \\
	\end{pmatrix}$かつ$L \in \Pic(X)$については
\begin{equation}
	\theta((-) \otimes L) =
	\begin{pmatrix}
		1      & 0 \\
		\deg L & 1 \\
	\end{pmatrix}
\end{equation}
となるため、$\theta(A(X)) \subsetneq \SL(2, \bZ)$と$A(X) \subsetneq \Auteq D^b(X)$がわかる。


\subsection{Fourier--向井変換と捻り関手}
Fourier--向井変換とは、積分核と呼ばれる$X \times Y$の導来圏の対象$P \in D^b(X \times Y)$から構成される関手$\Phi^P =\Phi^P_{X \to Y}\colon D^b(X) \to D^b(Y)$である。
$X, Y$を非特異射影多様体とし、$p_X \colon X\times Y \to X, p_Y \colon X \times Y \to Y$を射影とする。
このとき、$P \in D^b(X \times Y)$に対し関手
\begin{equation}
	\Phi^P_{X \to Y} \colon D^b(X) \xrightarrow{p_X^*} D^b(X \times Y) \xrightarrow{ - \otimes P} D^b(X \times Y) \xrightarrow{p_{Y*}} D^b(Y)
\end{equation}
を$P$を積分核とするFourier--向井変換と呼ぶ。
ここで$p_X^*, p_{Y*}, - \otimes P$はそれぞれ逆像、順像、テンソル積(の導来関手)を意味する。

以下の定理より、Fourier--向井変換は導来圏の間の関手を調べる上で基本的な道具となる。
\begin{theorem}[\cite{MR1465519}]
	$X, Y$を非特異射影多様体とし、$\Phi \colon D^b(X) \to D^b(Y)$
\end{theorem}
\section{小平ファイバーのミラー対称性と写像類群}
楕円曲面$\pi \colon S \to C$の特異ファイバーとして現れうる曲線は小平とNeronによりADE型のアファインDynkin図形を用いて分類されている。
これらの曲線を小平ファイバーと呼ぶ。
例えば$\rm{I}_n$型と呼ばれる曲線は$n$本の$\bP^1$が$n$角形のかたちに交わってできる特異曲線であり、アファイン$A_n$型Dynkin図形に対応する。



$Y_n$を$\rm{I}_n$型($n \geq 2$)の小平ファイバーとすると、そのミラー多様体は$n$点穴あきトーラス$T_n$であることが\cite{MR3663596}により示されている。
すなわち以下が成り立つ。
\begin{theorem}\cite{MR3663596}
	$\cW(T_n)$を$T_n$の巻深谷圏とすると、$\bC$線形三角圏の同値$D^b(Y_n) \cong D^b(\cW(T_n))$が存在する。
\end{theorem}
これにより、おおざっぱにいうと$Y_n$上の層とその間の$\Hom$空間の次元が、$T_n$上の曲線とその交差数に対応する。
より正確には、以下が成り立つ。
\begin{theorem}[\cite{2020arXiv201108288O}]
	\begin{enumerate}
		\item シフト関手$[1]$とホモトピーによる違いを除いて、以下のものが1体1対応する。\begin{itemize}
			      \item $D^b(Y_n)$の直既約対象$F$の同型類
			      \item $T_n$上の曲線$\gamma$とその上の直既約$\bC$局所系$V$の組
		      \end{itemize}
		\item (1)の対応で$E,F$と$(\gamma_E, V_E), (\gamma_E, V_F)$が対応し、$E$または$F$がperfectで$\dim V_E = \dim V_F = 1$を満たすとする。このとき\begin{equation}
			      \sum_{i}\dim\Ext^i(E, F) = \#(\gamma_E \cap \gamma_F)
		      \end{equation}が成り立つ。
	\end{enumerate}
\end{theorem}
さらにこの対応を用いると、$D^b(Y_n)$の自己同値群について
(楕円曲線の場合と似た)以下の記述が得られる。
\begin{theorem}\cite[Theorem D]{2020arXiv201108288O}
	完全列
	\begin{equation}
		1 \to \Aut^0(Y_n) \times \Pic^0(Y_n)\times \bZ[1] \to \Auteq{D^b(Y_n)} \xrightarrow{\Upsilon} \MCG(T_n) \to 1
	\end{equation}
	が存在する。
	ここで$\Upsilon$はミラー対称性による対応と以下の意味で整合的である。
	\begin{itemize}
		\item $\Phi \in \Auteq{D^b(Y_n)}$と直既約な$F \in D^b(Y_n)$について、$\Upsilon(\Phi)$は$F$に対応する曲線を$\Phi(F)$に対応する曲線に写す。
	\end{itemize}
\end{theorem}
さらに$\Upsilon$によって、球面対象$F$に付随する捻り関手$T_F$が曲線$\gamma_F$に付随するDehn捻りに写されることもわかる。
\section{半捻り関手}
\begin{theorem}
	$\pi \colon X \to T$を非特異凖射影多様体の間の平坦射とし、$i \colon Y = \pi^{-1}(0) \to X$をファイバーとする。
	また$E \in D^b(Y)$を、$i_*E \in D^b(X)$が球面対象になるような対象とする。
	このとき以下の図式を(自然同型の違いを除いて)可換にするような同値$H_E \colon D^b(Y) \to D^b(Y)$が一意に存在する。
	\begin{equation}
		\begin{tikzcd}
			D^b(Y) \arrow[r,"i_*"]\arrow[d, "H_E"'] & D^b(X) \arrow[d,"T_{i_*E}"]\\
			D^b(Y) \arrow[r, "i_*"]& D^b(X).
		\end{tikzcd}
	\end{equation}
\end{theorem}
\section{楕円曲面の自己同値群}
以上の半捻り関手を利用して、あるクラスの楕円曲面の自己同値群の記述を与えることができる。
まず\cite{MR3568337}による以下の結果がある。
\begin{theorem}[\cite{MR3568337}]
	\begin{itemize}
		\item $\pi \colon S \to C$を相対的極小な楕円曲面とし、
		\item 小平次元$\kappa(S)$は$0$ではなく、
		\item $\pi$の可約ファイバーは$\rm{I}_n$型($n \geq 2$)で重複ファイバーでないとする。
	\end{itemize}
	さらに部分群$B \subset \Auteq{D^b(S)}$を
	\begin{equation}
		B = \langle T_{\cO_G(a)} \mid G \subset S \text{: $(-2)$-曲線, } a \in \bZ \rangle
	\end{equation}
	と定義する。
	このとき完全列
	\begin{align}
		1 \to \langle B, (-)\otimes \cO_S(D)\mid D.F=0, F \textrm{: ファイバー} \rangle & \rtimes \Aut{S} \times \bZ[2]                      \\
		                                                                                & \to \Auteq{D^b(S)} \xrightarrow{\Theta} \SL(2,\bZ)
	\end{align}
	が存在する。
\end{theorem}
\cite{MR3568337}ではさらに$\Theta$の像の特徴づけも与えてられている。
よってこのような楕円曲面に対しては$\Auteq D^b(S)$の構造は$B$の構造の研究に帰着されることになる。
\begin{theorem}
	$\pi \colon S \to C$の可約ファイバー全体を$Y_{n_1}, \dots, Y_{n_l}$とする。
	ここで$Y_{n_j}$は$\rm{I}_{n_j}$型だとする。
	このとき次の完全列が存在する。
	\begin{equation}
		1 \to \langle (-)\otimes \cO_S(Y_{n_j}) \mid j = 1, \dots, l\rangle \to B \to \prod_{j=1}^l \MCG(T_{n_j}).
	\end{equation}
\end{theorem}
\bibliographystyle{amsalpha}
\bibliography{half_twist}
\end{document}