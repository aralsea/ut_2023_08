\documentclass[uplatex,a4paper,dvipdfmx]{jsarticle}


\usepackage{amsfonts, amsmath, amsthm, amssymb}
\usepackage{mathtools}
\renewcommand{\theenumi}{\arabic{enumi}}
\renewcommand{\labelenumi}{(\theenumi) }
\providecommand{\MR}{\relax\ifhmode\unskip\space\fi MR }
%% Theorems
\theoremstyle{plain}

\newtheorem{theorem}{Theorem}[section]
\newtheorem{conjecture}[theorem]{Conjecture}
\newtheorem{lemma}[theorem]{Lemma}
\newtheorem{proposition}[theorem]{Proposition}
\newtheorem{corollary}[theorem]{Corollary}

\theoremstyle{definition}
\newtheorem{definition}[theorem]{Definition}
\newtheorem{example}[theorem]{Example}
\newtheorem{remark}[theorem]{Remark}

%% Some operators
\DeclareMathOperator{\Hom}{\mathrm{Hom}}
\DeclareMathOperator{\Tor}{\mathrm{Tor}}
\DeclareMathOperator{\CHom}{\mathcal{H}\!\mathit{om}}
\DeclareMathOperator{\CTor}{\mathcal{T}\!\mathit{or}}
\DeclareMathOperator{\Auteq}{\mathrm{Auteq}}
\DeclareMathOperator{\Cone}{\mathrm{Cone}}
\DeclareMathOperator{\ev}{\mathrm{ev}}
\DeclareMathOperator{\id}{\mathrm{id}}
\DeclareMathOperator{\depth}{\mathrm{depth}}
\DeclareMathOperator{\Pic}{\mathrm{Pic}}
\DeclareMathOperator{\MCG}{\mathrm{MCG}}
\DeclareMathOperator{\PMCG}{\mathrm{PMCG}}
\DeclareMathOperator{\RHom}{\mathrm{RHom}}
\DeclareMathOperator{\Ker}{\mathrm{Ker}}
\DeclareMathOperator{\Image}{\mathrm{Im}}
\DeclareMathOperator{\Aut}{\mathrm{Aut}}
\DeclareMathOperator{\Inn}{\mathrm{Inn}}
\DeclareMathOperator{\Out}{\mathrm{Out}}
\DeclareMathOperator{\Supp}{\mathrm{Supp}}
\DeclareMathOperator{\SL}{\mathrm{SL}}
\DeclareMathOperator{\Spec}{\mathrm{Spec}}
\DeclareMathOperator{\Perf}{\mathrm{Perf}}
\DeclareMathOperator{\NS}{\mathrm{NS}}
\DeclareMathOperator{\Ext}{\mathrm{Ext}}
\DeclareMathOperator{\Hilb}{\mathrm{Hilb}}
\DeclareMathOperator{\res}{\mathrm{res}}
\DeclareMathOperator{\Ch}{\mathrm{Ch}}
\DeclareMathOperator{\coh}{\mathrm{coh}}


\newcommand{\nc}{\newcommand}

%% Calligraphic letters

\nc{\cF}{{\mathcal{F}}}
\nc{\cG}{{\mathcal{G}}}
\nc{\cH}{{\mathcal{H}}}

\nc{\cO}{{\mathcal{O}}}
\nc{\cU}{{\mathcal{U}}}
\nc{\cW}{{\mathcal{W}}}

%% Blackboard letters
\nc{\bA}{{\mathbb{A}}}
\nc{\bC}{{\mathbb{C}}}
\nc{\bP}{{\mathbb{P}}}
\nc{\bQ}{{\mathbb{Q}}}
\nc{\bZ}{{\mathbb{Z}}}


\title{代数多様体の導来圏の半捻り関手}
\author{荒井 勇人}
\date{\today}
\begin{document}
\maketitle

\begin{abstract}
	代数多様体$X$の連接層の導来圏$D^b(X)$の自己同値を構成する重要な手法のひとつが、SeidelとThomasにより導入された球面対象$E \in D^b(X)$に沿う捻り関手$T_E \in \mathrm{Auteq} D^b(X)$である。これはホモロジー的ミラー対称性のもとでシンプレクティック多様体上のLagrange球面に沿うDehn捻りに対応する。

	本講演では代数多様体の平坦族$X \to T$について、適切な設定のもと$D^b(X)$上の捻り関手をファイバー$X_t$の導来圏の自己同値に『制限』できるという結果について紹介する。
	特にその具体例として楕円曲面と可約ファイバーの場合を考えることで、楕円曲面の自己同値群を穴あきトーラスの写像類群の言葉で記述できることを説明する。
	さらにこれらの具体例から、『制限』によって得られる導来圏の自己同値が実曲面上の弧(arc)に沿った半捻り(half twist)のミラー対称性による類似物であると考えられること、およびこの現象に関連した今後の展望についても述べたい。
\end{abstract}

\section{導入}
\subsection{代数多様体の導来圏と自己同値群}
$X$を$\bC$上の代数多様体とし、$X$上の連接層のなすアーベル圏を$\coh X$、その有界な(コチェイン)複体のなすアーベル圏を$\Ch^b(\coh X)$とする。
複体の間の射$f \colon E \to F$は、複体のコホモロジーに誘導する射$H^n(f) \colon H^n(E) \to H^n(F)$が全ての$n \in \bZ$について同型であるとき擬同型と呼ばれる。
$X$の導来圏$D^b(X)$とは、$\Ch(\coh X)$を擬同型全体で局所化して得られる圏であり、自然に$\bC$線形な三角圏の構造を持つ。
また$D^b(X)$の自己同値群$\Auteq D^b(X)$を、$\bC$線形で三角圏の構造を保つような$D^b(X)$の自己同値関手の同型類がなす群とする。
$D^b(X)$の自己同値として、$X$の自己同型$f$による連接層の引き戻し$f^*$、$X$上の直線束$L$によるテンソル積$(-)\otimes L$、および三角圏のシフト関手$(-)[1]$が常に存在し、これらの生成する$\Auteq D^b(X)$の部分群
\begin{equation}
	A(X) \coloneq \Aut(X) \ltimes \Pic(X) \times \bZ[1]
\end{equation}
の元は標準的な自己同値と呼ばれる。
自己同値群についてのもっとも基本的な結果は次のBondalとOrlovによる定理である。
\begin{theorem}[\cite{MR1818984}]
	$X$を非特異射影多様体とし、$X$の標準束$\omega_X$またはその双対束$\omega_X^\vee$が豊富な直線束だと仮定する。
	このとき$A(X) = \Auteq D^b(X)$である。
\end{theorem}
この結果により、非自明な包含関係$A(X) \subsetneq \Auteq D^b(X)$が生じうる一番単純な状況は$X$が楕円曲線の場合となる。
このときの$\Auteq D^b(X)$はOrlovによるアーベル多様体についての結果を適用すると以下のようになる。
\begin{theorem}[\cite{MR1921811}]
	$X$を楕円曲線とすると、以下の完全列が存在する。
	\begin{equation}\label{eq:autoequivalence_of_elliptic_curve}
		1 \to \Aut{X} \ltimes \Pic^{0}(X)\times \bZ[2] \to \Auteq{D^b(X)} \xrightarrow{\theta} \SL(2, \bZ) \to 1
	\end{equation}
	ここで$\theta$は偶数次のコホモロジー群$H^0(X, \bZ) \oplus H^0(X, \bZ) \cong \bZ^2$への作用で与えられる。
\end{theorem}
これにより$A(X)$の像を計算すると、$\theta((-)[1]) = \begin{pmatrix}
		-1 & 0  \\
		0  & -1 \\
	\end{pmatrix}$かつ$L \in \Pic(X)$については
\begin{equation}
	\theta((-) \otimes L) =
	\begin{pmatrix}
		1      & 0 \\
		\deg L & 1 \\
	\end{pmatrix}
\end{equation}
となるため、$\theta(A(X)) \subsetneq \SL(2, \bZ)$と$A(X) \subsetneq \Auteq D^b(X)$がわかる。


\subsection{Fourier--向井変換と捻り関手}
多様体の導来圏の間の関手を構成する一般的な方法であるFourier--向井変換と、自己同値の重要な構成法である捻り関手について説明する。

$X, Y$を非特異射影多様体とし、$p_X \colon X\times Y \to X, p_Y \colon X \times Y \to Y$を射影とする。
このとき、$D^b(X \times Y)$の対象$P$をとるごとに関手$\Phi^P_{X \to Y}$
\begin{equation}
	\Phi^P_{X \to Y}(-) \coloneqq p_{Y*}(p_X^*(-) \otimes P)\colon D^b(X) \to D^b(Y)
\end{equation}
\section{小平ファイバーのミラー対称性と写像類群}

\section{半捻り関手}

\section{楕円曲面の自己同値群}


\bibliographystyle{amsalpha}
\bibliography{half_twist}
\end{document}